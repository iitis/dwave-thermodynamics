% ****** Start of file apssamp.tex ******
%
%   This file is part of the APS files in the REVTeX 4.2 distribution.
%   Version 4.2a of REVTeX, December 2014
%
%   Copyright (c) 2014 The American Physical Society.
%
%   See the REVTeX 4 README file for restrictions and more information.
%
% TeX'ing this file requires that you have AMS-LaTeX 2.0 installed
% as well as the rest of the prerequisites for REVTeX 4.2
%
% See the REVTeX 4 README file
% It also requires running BibTeX. The commands are as follows:
%
%  1)  latex apssamp.tex
%  2)  bibtex apssamp
%  3)  latex apssamp.tex
%  4)  latex apssamp.tex
%
\documentclass[%
 reprint,
%superscriptaddress,
%groupedaddress,
%unsortedaddress,
%runinaddress,
%frontmatterverbose, 
%preprint,
%preprintnumbers,
%nofootinbib,
%nobibnotes,
%bibnotes,
 amsmath,amssymb,
 aps,
%pra,
%prb,
%rmp,
%prstab,
%prstper,
%floatfix,
]{revtex4-2}

\usepackage{graphicx}% Include figure files
\usepackage{dcolumn}% Align table columns on decimal point
\usepackage{bm}% bold math
\usepackage{hyperref}% add hypertext capabilities
%\usepackage[mathlines]{lineno}% Enable numbering of text and display math
%\linenumbers\relax % Commence numbering lines

%\usepackage[showframe,%Uncomment any one of the following lines to test 
%%scale=0.7, marginratio={1:1, 2:3}, ignoreall,% default settings
%%text={7in,10in},centering,
%%margin=1.5in,
%%total={6.5in,8.75in}, top=1.2in, left=0.9in, includefoot,
%%height=10in,a5paper,hmargin={3cm,0.8in},
%]{geometry}

\begin{document}

%\preprint{APS/123-QED}

\title{Energetic Efficiency of Quantum Annealers}% Force line breaks with \\
%\thanks{A footnote to the article title}%

\author{Tomasz Śmierzchalski}
\email{tsmierzchalski@iitis.pl}
 %\altaffiliation[Also at ]{Physics Department, XYZ University.}%Lines break automatically or can be forced with \\
\author{Zakaria Mzaouali}%
 \email{zmzaouali@iitis.pl}
 \author{Bartłomiej Gardas}
 \email{bgardas@iitis.pl}
\affiliation{%
Institute of Theoretical and Applied Informatics, Polish Academy of Sciences, Bałtycka 5, Gliwice, 44-100, Poland.
}

\date{\today}% It is always \today, today,
             %  but any date may be explicitly specified

\begin{abstract}
We study the thermodynamical efficiency of D-Wave quantum annealers subject to reverse and reverse annealing protocols. We show the trade-off between computational precision and its energy cost

\end{abstract}

%\keywords{Suggested keywords}%Use showkeys class option if keyword
                              %display desired
\maketitle

\section{\label{intro} Introduction}

\section{\label{theory} Theory}
The D-wave processor is an open quantum system that exchanges energy in the form of heat with the environment and work with the external time-dependent control fields. Accordingly, the D-wave processor is a thermal machine which was recently confirmed,  where it was shown that the D-Wave 2000Q under the reverse-annealing protocol, operates as a quantum heat engine.  \\
Motivated by the recent results for high fidelity control of the ground state of the Ising model by minimising the adiabatic action, we are interested in analyzing the thermodynamical efficiency of D-Wave quantum annealers, via different quantum annealing schedules: reverse, and pausing. 

The Hamiltonian governing the evolution of the D-Wave chip is:
\begin{equation}
    H(s_t)=(1-s_t)\sum_i \sigma_i^x + s_t \left( \sum_i h_i \sigma_i^z + \sum_{\langle i,j \rangle} J_{i,j} \sigma_i^z \sigma_j^z \right).
    \label{ham}
\end{equation}

We assume that initially the system+environment state is given by:
\begin{equation}
    \rho=\frac{\exp{(-\beta_1 H_S)}}{Z_S} \otimes \frac{\exp{(-\beta_2 H_E)}}{Z_E}.
\end{equation}

Accordingly, the quantum exchange fluctuation theorem gives:

\begin{equation}
    \frac{p(\Delta E_1,\Delta E_2)}{p(-\Delta E_1,-\Delta E_2)}=e^{\beta_1 \Delta E_1 + \beta_2 \Delta E_2},
    \label{eq3}
\end{equation}
Where $\Delta E_i, i=1,2$ are, respectively, the (stochastic) energy changes of the processor and its environment, occurring in the scheduled time $\tau$, and $p(\Delta E_1,\Delta E_2)$ is the joint probability of their occurrence in a single run of the annealing schedule.\\
By identifiying the entropy production $\Sigma =\beta_1 \Delta E_1 + \beta_2 \Delta E_2$. Eq.~\eqref{eq3} can be re-written as:
\begin{equation}
    \frac{p(\Sigma,\Delta E_1)}{p(-\Sigma,-\Delta E_1)}=e^{\Sigma}. 
\end{equation}
Using the thermodynamic uncertainty relation, we can express lower bounds of the entropy production $\langle \Sigma \rangle$, the average work $\langle W\rangle$, and the average heat $\langle Q\rangle$ as a function of the energy change of the processor $\Delta E_1$:
\begin{align}
    \langle \Sigma \rangle &\geq  2g\left(\frac{\langle \Delta E_1 \rangle}{\sqrt{\langle \Delta E_1^2 \rangle}}\right),\\
    -\langle Q \rangle &\geq \frac{2}{\beta_2}g\left(\frac{\langle \Delta E_1 \rangle}{\sqrt{\langle \Delta E_1^2 \rangle}}\right) - \frac{\beta_1}{\beta_2} \langle \Delta E_1 \rangle,\\
    \langle W \rangle &\geq \frac{2}{\beta_2}g\left(\frac{\langle \Delta E_1 \rangle}{\sqrt{\langle \Delta E_1^2 \rangle}}\right) + \left( 1 - \frac{\beta_1}{\beta_2} \right) \langle \Delta E_1 \rangle,
\end{align}
where $g(x)=x\tanh^{-1}{(x)}$.

The thermodynamic efficiency is given by 
\begin{equation}
    \eta_{\text{th}} \leq - \frac{ \langle W \rangle}{ \langle Q \rangle}.
\end{equation}
\section{\label{Experiments} Experiments}

\section{\label{conclusion} Conclusion}

\section{Acknoweldgements}
The authors acknowledge support from the National Science Center (NCN), Poland, under Project No.~2020/38/E/ST3/00269.

\bibliography{bib}
\bibliographystyle{abbrv}
\end{document}
%
% ****** End of file apssamp.tex ******
